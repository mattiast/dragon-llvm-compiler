\RequirePackage[l2tabu, orthodox]{nag}
% rubber: module pdftex
\documentclass[12pt]{article}
\usepackage[utf8]{inputenc}
\usepackage[T1]{fontenc}
\usepackage[finnish]{babel}
\usepackage{amsfonts,amsmath,amssymb,amsthm,booktabs,enumitem,graphicx,url}
\usepackage[numbers,sort&compress]{natbib}
\usepackage{lmodern}
\usepackage[top=2cm]{geometry}

\usepackage{microtype} % This must be after fonts
\usepackage[colorlinks=false, pdfborder={0 0 0}]{hyperref} % hyperref and cleveref must be loaded last
\usepackage{cleveref}

\newcommand{\code}{\texttt}

\title{Ohjelmointikielten kääntäjät \\ Lopputyö}
\author{Matti Åstrand}
\date{28. 5. 2008}

\begin{document}
\maketitle

\section{Johdanto}
Harjoitustyönä toteutettiin kääntäjä oppikirjan \cite{dragon} kielestä LLVM-tavukoodiin. 
Kääntäjä kirjoitettiin Haskell-kielellä.
Kieli toteutettiin kokonaan yhtä poikkeusta lukuunottamatta 
 (katso kohta \ref{kuva}).

\section{Kääntämisohjeet}
Ohjelman voi kääntää \code{Makefile}:n avulla. Komennolla \code{make} ohjelma 
käännetään, ja testit ajetaan komennolla \code{make test}. Nämä toimivat TKTL:n 
järjestelmässä, koska LLVM:n binäärit löytyvät hakemistosta \code{/opt/llvm-2.1/bin}. 
Jos testejä halutaan ajaa muualla, tulee \code{Makefile}:ä muuttaa vastaavasti.

\section{Työn kuvaus}
\subsection{Kieli}
Oppikirjan kielessä ohjelma on vain yksi lohkorakenne, joka voi sisältää 
\code{while}-silmukoita ja \code{if-else}-rakenteita. Esimerkiksi funktioita 
tai proseduureja ei siis ole. Kielessä on tyyppeinä totuusarvot (\code{bool}), 
kokonaisluvut, liukuluvut sekä merkit (\code{char}) sekä näista muodostetut 
vakiokokoiset taulukot. Erikoista on, ettei \code{char}-tyypin muuttujaan 
voi sijoittaa mitään, sillä syntaksista ei löydy tähän sopivaa lauseketta.

Kielessä voi muodostaa lausekkeita seuraavilla kaksipaikkaisilla 
operaattoreilla: 
\code{+ - * / == != < <= > >= \&\& ||}, sekä yksipaikkaisilla operaattoreilla: 
\code{- !}. Seuraavassa taulukossa on kuvattu, miten lausekkeen tyyppi 
riippuu argumenttien tyypeistä:

\begin{tabular}{ccl}
\code{int + int     }& = & \code{int  } (samoin \code{-, *, /})\\
\code{int + float   }& = & \code{float} \\
\code{float + int   }& = & \code{float} \\
\code{float + float }& = & \code{float} \\
\code{int < int     }& = & \code{bool } (samoin \code{>, ==} jne.)\\
\code{float < float }& = & \code{bool }\\
\code{bool \&\& bool}& = & \code{bool } (samoin \code{||})\\
\code{- int         }& = & \code{int  }\\
\code{- float       }& = & \code{float} \\
\code{!bool         }& = & \code{bool }\\
\end{tabular} 

\subsection{Jäsentäjä}\label{kuva}
Jäsentäjä toteutettiin Haskellin Parsec-kirjastoa. Kirjasto toimii siten, että 
yksin\-kertaisia jäsentäjiä yhdistellään isommiksi tavalla, joka vastaa kielen 
määrittelyä. Jäsentäjä on tiedostossa \code{Parser.hs}.

Lausekkeiden jäsentäjään käytettiin Haskellin valmista 
\code{buildExpressionParser}-funktiota. Tämän vuoksi jäsentäjässä on sellainen
rajoite, että yksipaikkaisia operaattoreita ei voi olla montaa peräkkäin, 
kuten lausekkeissa \code{!!(x==x)} ja \code{-{}-x}. Onneksi molemmat mahdolliset 
tapaukset ovat käytännössä hyödyttömiä, sillä ne ovat täysin yhtäpitäviä 
lausekkeiden \code{x==x} ja \code{x} kanssa. Jos välttämättä halutaan käyttää 
ylläolevia lausekkeita, niihin on laitettava sulut seuraavasti: \code{!(!(x==x)), -(-x)}.

\subsection{Tyypintarkistus}
Tyypintarkistuksessa pitää huolehtia, että silmukoiden ja \code{if}-rakenteiden 
testilausekkeet ovat \code{bool}-tyyppisiä ja että sijoituksessa lausekkeen ja 
muuttujan tyypit ovat yhteensopivia. Tässä pitää ottaa huomioon, että \code{int}-lausekkeen
voi sijoittaa \code{float}-muuttujaan ja päinvastoin. Tähän käytettiin LLVM:n 
käskyjä \code{sitofp} ja \code{fptosi}. Tyypintarkistin on tiedostossa \code{Static.hs}.
Toteutuksessa luo\-daan ensin symbolitaulut, jossa on jokaisen muuttujan tyyppi. 
Sitten lausekkeiden tyypit voi määrittää ylläolevan taulukon säännöillä.

\subsection{Koodin generointi}
Lausekkeiden laskeminen onnistuu LLVM:n käskyillä helposti, sillä jokaiselle 
kaksipaikkaiselle operaattorille löytyy käsky. Yksipaikkaiset operaattorit 
toteutettiin siten, että \code{-x} muutetaan \code{(0-x)}:ksi ja 
\code{!b} muutetaan \code{(x XOR true)}:ksi.

Ennen kuin koodia aletaan generoida, valitaan jokaiselle muuttujalle yksi\-käsitteinen 
nimi, joka on LLVM-koodissa osoitemuuttuja. Esimerkiksi \code{int}-muuttujaa vastaa 
rekisteri, joka on tyyppiä \code{i32*}. Muuttujien nimet eivät riipu alkuperäisistä 
nimistä, vaan ovat muotoa \code{ \%var1, \%var2} \ldots

Laskentaan tarvitaan paljon rekistereitä välituloksille, joten ohjelmassa pidetään 
yllä tilaa käytetyistä rekistereistä. Ongelmaksi muodostui \code{break}-käskyn toteuttaminen, 
sillä tällöin pitää ohjelmassa hypätä käsiteltävän rakenteen ulko\-puolelle. Tätä varten 
pitää ohjelmassa ylläpitää tilaa myös sisäkkäisistä \code{while}-silmukoista, jotta tiedetään mihin 
pitää milloinkin hypätä. 

Yksi ongelma oli LLVM:n sääntö, että jokaisen peruslohkon pitää päättyä hyppykäskyyn, eli 
ohjelma ei voi vain valua seuraavaan peruslohkoon. Tämä aiheutti joukon virhetilanteita, 
jotka vältettiin lisäämällä koodin generoinnissa ylimääräisiä hyppykäskyjä, kunnes kaikki 
testit menivät läpi hyväksytysti.

Koodin generoija on tiedostossa \code{CodeGen.hs}.

\section{Testaus}
Kääntäjän testaukseen käytettiin oppikirjan valmiita testejä hieman muutettuina. 
Testiohjelmiin tehtiin seuraavat muutokset:
\begin{itemize}
\item Testiohjelmissa \code{jump1.t} ja \code{prog4.t} esiintyi käskynä pelkkä puolipiste, 
joka ei ole kielen määrittelyn mukainen ja joka tuottaa syntaksivirheen. Tämä käsky korvattiin 
tyhjällä lohkorakenteella \code{\{\}}, joka tuottaa halutun lopputuloksen.

\item Ohjelmassa \code{expr3.t} esiintyy lausekkeita, joissa on kaksi negaatiota peräk\-käin, kuten 
kohdassa \ref{kuva}. Nämä lausekkeet sulutettiin edellä mainitulla tavalla. 

\item Ohjelmat \code{expr4.t, jump3.t} ja \code{prog0.t} kaatuivat ajonaikana, koska niissä 
viitataan taulukon \code{i}:n alkioon, vaikka muuttujaa \code{i} ei ole alustettu ja se 
sisältää satunnaista roskaa. Näihin ohjelmiin lisättiin muuttujien alustuksia.

\item Ohjelmat \code{prog0.t, prog1.t} ja \code{identity2.t} jäivät ikuiseen silmukkaan ajettaessa. 
Näitä muutettiin niin, että ne pysähtyvät.

\item Ohjelmassa \code{expr1.t} esiintyy liukulukuvakio 3.14159. LLVM vaatii, että vakiot ovat 
esitettävissä tarkasti binäärilukuina, joten vakio muutettiin 3.140625:ksi, joka on 
binäärimuodossa tarkasti 11.001001. Tämän muutoksen olisi voinut toteuttaa automaattiseksi suoraan 
kääntäjään, mutta sitä ei toteutettu. 
\end{itemize}

\section{Parannusmahdollisuuksia}
Kääntäjää voidaan parantaa ainakin seuraavilla tavoilla:
\begin{itemize}
\item Toteuttaa moninkertaiset negaatiot ja vastaluvut jollain omalla lisäyksellä 
jäsentäjään

\item Muuttaa liukulukuvakiot automaattisesti LLVM:n haluamaan muotoon, eli niin, että 
ne voidaan tarkasti esittää binäärimuodossa

\item Lisätä tietorakenne LLVM:n tavukoodin esittämiseen. Tämä johdonmukais\-taisi koodin 
generointia, kun ei tarvitse joka välissä muistaa esim. rivinvaihtoja käskyjen väliin. 
Tällä hetkellä kääntäjä tuottaa syntaksipuista suoraan merkkijonoja. Tällainen tietorakenne 
on välttämätön myös, jos tuotettua koodia halutaan optimoida. 
\end{itemize}

\begin{thebibliography}{9}
\bibitem{dragon}
  A. V. Aho, M. S. Lam, R. Sethi, and J. D. Ullman,
  Compilers: \emph{Principles, Techniques and Tools}, 
  Addison Wesley,
  2nd Edition,
  2006.
\end{thebibliography}
\end{document}
